% ************************** Thesis Abstract *****************************
% Use `abstract' as an option in the document class to print only the titlepage and the abstract.
\begin{abstract}
Le travail accompli durant cette these a consiste en l’investigation d’une solution informatique permettant de garantir la reproductibilite de resultats scientifiques d’origine informatique ou experimentale. Lorsque nous utiliserons le mot ‘reproductibilite’ sans precision supplementaire nous entendons par la l’obtention de resultats dans la marge de precision acceptable du domaine scientifique concerne. Aussi nous entendons par ‘garantir’ ici que la solution dans le defaut de reproduire des resultats dans un environnement donne pourra proposer clairement les informations concernant l’infrastructure requise pour pouvoir realiser les resultats attendus.
Au moment meme de cette redaction, en ce qui concerne la reproductibilite de resultats scientifiques de nature informatique ou experimentale contenu generalement dans une publication, plusieurs obstacles jalonnent le chemin vers leurs reconstructions. Ces obstacles sont de plusieurs nature: techniques, methodologiques et communautaires. Or la reproductibilite est la pierre angulaire de la methode scientifique parce qu'elle garantie les fondements sur lesquels de nouveaux savoirs se forment et sont valides puis accepter dans la communaute. Elle aussi le garant de la solidite des fondements sur lequel peut s’etablir avec assurance la continuite de nouvelles recherches entreprises pour l’avancement de la Science. 
Dans la recherche scientifique, la communaute est une ressource formidable pour collaborer, apprendre, partager, verifier, valider et contribuer a la Science. C’est le centre de toute les activites scientifiques d’un domaine precis. Par consequent, nous avons donc juge evident qu’une solution unifiante devrait commencer par federer les chercheurs en developpant un sens communautaire du partage, de la collaboration et de l’interoperabilite autour de resultats reproductibles. En effet, il est important de voir que la communaute est le noyau autour duquel les resultats sont enrichis par les diverses collaborations. Et cet enrichissement beneficie les chercheurs contributeurs autant que ceux qui sont plus interesses par des procedes en amont sur ces resultats enrichis.
{Constat rapide de la situation et l’importance}
Sur la question de la reproductibilite des resultats de recherche, nous nous sommes interesses aux deux principales approches scientifiques du 21em siecle: l’experimentation pure, la modelisation informatique mais aussi leur intersection. Par experimentation nous entendons ici l’utilisation d’une machine specialisee pour produire ou analyser des donnees relative a une investigation physiquement mise en place. Aussi par modelisation informatique nous faisons references aux investigations complement simulees sur un ou un ensemble d’ordinateurs. Quant meme que le soucis de reproduire ai ete inherent chez les experimentalistes a travers des favoris comme le bloc-note de laboratoire, les notes prises etaient trop personalisees et generalement insuffisantes ou trop endomagees pour etre d’une utilite quelconque a d’autres chercheurs utilisant la meme machine et encore moins un chercheur utilisant une autre machine. Ces experiences donnent aujourd'hui un sens important a la necessite de representations communes de l’information recolte pendant une investigation pour qu’elle soit comprehensible entre chercheurs mais egalement entre machines. Un autre objectif capital est la nature cruciale de l’automatisation de l’execution des investigations du fait du facteur manuel dans les actions des chercheurs qui pese sur l’incertitude des resultats. De plus la complexite meme grandissante des systemes de nos jours rend le bloc note obsolete parceque le nombre de parametres a ecrire et entraine une divergence plus prononcee loin de nos espoirs d’inter-operabilite entre machines. En outre, meme si le soucis de la reproductibilite des resultats informatiques aient emerges plus tardivement dans les annees 60 a travers les efforts de Chercheurs comme Jon Claerbout, la conception meme des systemes informatiques fait que certains problemes souleves dans le monde experimental ne se posent pas ici. Neanmoins, des obstacles d’autres natures existent. La formalisation de l’information utile pour reproduire une execution conduisant aux ‘memes’ resultats restent un challenge du fait que l’objectif est de pouvoir reutiliser cette information pour reproduire une execution le plus simplement et automatiquement possible. Aussi la particularite virtuelle meme de la modelisation informatique laissant entrevoir par exemple un monde sans bornes physiques concretes (a l’exception des limites materiels) ouvrent la porte a un domaine supplementaire d’erreurs propres au milieu informatique. De ce fait d’autres problemes lies meme a la conception materielle et logicielle des ordinateurs ne favorisent pas la reproductibilite par defaut des executions. Et quand a la complexite des architectures nous sommes entrain aujourd'hui de concretement aller a l’ExaScale (1 milliard de milliards de calculs a la seconde). Des systemes dont la dimension et la complexite soulevent d’autres problemes en plus que ceux deja cite. Les divers defis exposes ici pour ces deux approches presentent des perspectives cruciales d’un point de vue de consensus dans la determination et la representation de l’information necessaire lors d’une investigation scientifique de maniere reproductible et aussi de formes de collaborations et de partages favorisant l’automatisation et l’interoperabilite. Par ailleurs, notre recherche pour la perennite meme des propositions exposes, doit regarder au futur de l’experimentation et de l’informatique et y etre deja adapte: Les laboratoratories integres et robotises et Les infrastructures ExaScale.
{methodology in brief}
Dans le cadre de cette these, l’etat de l’art s’est organise independamment des approches scientifiques exposees. Nous avons identifie trois aspects fondamentaux a l’etablissement unifie d’une solution pour la reproductibilite des resultats de recherches: La structure, le contenu et le modele d’enregistrement des informations utiles lors d’une investigation; Le partage, la collaboration, l’inter-oprabilite et l’automatisation a partir des enregistrements; L’acces, l’extraction des enregistrements enrichis par des collaborations et les differentes applications possibles. Plus en profondeurs nous exposons les problemes rencontres dans ces aspects, les solutions a ces problemes ainsi que les problemes dans ces solutions. Nous considererons l'environnement actuel des approches puis nous detaillons en perspectives nos attentes et nos predictions concernant le futur des environnements vis a vis des deux approches scientifiques. La proposition que nous exposons est une solution d’unification qui tente de resoudre le probleme de la reproductibilite dans les investigations scientifiques. Elle presente la necessite de consensus et de solutions qui tentent de resoudre les problemes rencontres aussi bien en informatique qu’en experimentation en essayant au mieux d’avoir des approches generiques unificatrices. Elle aborde chacun des trois aspects identifies en prenant en compte l’etat actuel de la technologie mais egalement la direction technologique dans les 50 prochaines annees a venir.
{Results of propositions} ** Next work here **
Ainsi, l’implementation de notre solution a travers la proposition d’une interface standard minimal de  communication (ISMC) pour les machines et un model d’enregistrement integrant jusqu’aux executions sur les clusters nous a permis  de garantir un certain nombre de situations. Dans le cas de l’experimentation, l’application sur un prototype de machine integree pour des experiences en BioScience nous a permis d’assurer la repetition et de verifier la reproduction des investigations. Par ailleurs aussi dans l’aspect informatique nous avons ete capables de tester l’enregistrement, le repetition et la verification de la reproductibilite d’investigation aussi bien sur des applications sequentielles, paralleles et distribuees sur cluster a noeuds homogenes et heterogenes. Cette derniere nous permettant de garantir l’aspect ExaScale pret de notre travail. Il est fondamental egalement d’ajouter que l’application de la partie communautaire en terme de collaboration, de partage et d’enrichissement des enregistrements, nous sommes arrives a une platform cloud suffisament generique pour etre compatible a tous les deux formes d’enregistrements possibles (experimental et informatiques). Cette platform vient egalement avec des outils permettant de connecter aussi bien les machines ayant implementees l’ISMC que les ordinateurs pour une automatisation et une inter-operabilite tres apreciable.
\end{abstract}
